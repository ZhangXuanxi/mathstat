\documentclass{article}% ctex 支持中文时可替换为 ctexart
\input{~/code/math_commands.tex}

% --- 开关:默认关闭 ---
\newif\ifwithrefs
\withrefsfalse
% 如果需要引用文献,把上面改成 \withrefstrue

\title{\huge R6 \ \normalsize}
\author{Xuanxi Zhang}
\begin{document}
\maketitle
\section{}
Suppose random variable  $X\sim \gU[0,\theta]$, where $\theta>0$ is an unknown parameter. $\{X_1,X_2,\ldots,X_n\}$ are i.i.d. samples drawn from the distribution of $X$. Consider two estimators of $\theta$:
\[\hat{\theta}_1=2\bar{X},\quad \hat{\theta}_2=\max_i X_{i},\]
where $\bar{X}=\frac{1}{n}\sum_{i=1}^n X_i$ is the sample mean.

\begin{enumerate}
    \item Find the bias and variance of $\hat{\theta}_1$ and $\hat{\theta}_2$.
    \item Find the mean squared error (MSE) of $\hat{\theta}_1$ and $\hat{\theta}_2$.
    \item Which estimator would you prefer? Explain your answer.
    \item Can we apply the cramer-rao lower bound to this problem? Why or why not?
\end{enumerate}

\section{Maximum a Posteriori}
$\{X_1,X_2,\ldots,X_n\}$ are i.i.d. samples drawn from $\gN(\theta,1)$.
\begin{enumerate}
    \item what is the maximum likelihood estimator (MLE) of $\theta$?
    \item If we have a prior distribution $\theta\sim \gN(\theta_0,\sigma^2)$, what is the maximum a posteriori (MAP) estimator of $\theta$?
\end{enumerate}

\section{likelihood ratio test}
$X=x$ is a single observation with PDF $f_\theta(x)=\theta x^{\theta-1}, \quad 0<x<1$

Find the most powerful level- $\alpha$ (level 5\%) test for
$$
\begin{aligned}
& H_0: \theta=3 \\
& H_1: \theta=2
\end{aligned}
$$


\ifwithrefs
  \bibliographystyle{plain}
  \bibliography{~/code/refs} % 修改为你的 bib 文件
\fi

\end{document}