\documentclass[11pt]{article}
\usepackage[a4paper,margin=1in]{geometry}
\usepackage{amsmath, amssymb, amsthm, mathtools}
\usepackage{bm}
\usepackage{enumitem}
\usepackage{booktabs}
\usepackage{hyperref}

\newtheorem{definition}{Definition}
\newtheorem{proposition}{Proposition}

\newcommand{\Pbb}{\mathbb{P}}
\newcommand{\E}{\mathbb{E}}
\newcommand{\Var}{\mathrm{Var}}
\newcommand{\1}{\mathbf{1}}

\title{Probability Review Notes}
\author{}
\date{}

\begin{document}
\maketitle

\section{Axioms of Probability (Kolmogorov)}
Let $(\Omega,\mathcal{F},\Pbb)$ be a probability space where $\Omega$ is the sample space, $\mathcal{F}$ a $\sigma$-algebra, and $\Pbb:\mathcal{F}\to[0,1]$ a probability measure. For all events $A,B\in\mathcal{F}$:
\begin{enumerate}[label=(\alph*)]
  \item \textbf{Nonnegativity:} $\Pbb(A)\ge 0$.
  \item \textbf{Normalization:} $\Pbb(\Omega)=1$.
  \item \textbf{$\sigma$-additivity:} If $A_1,A_2,\dots$ are pairwise disjoint, then
  $\Pbb\!\left(\bigcup_{i=1}^\infty A_i\right) = \sum_{i=1}^\infty \Pbb(A_i)$.
\end{enumerate}
\textit{Consequences:} $\Pbb(\varnothing)=0$; $\Pbb(A^c)=1-\Pbb(A)$; if $A\subseteq B$ then $\Pbb(B\setminus A)=\Pbb(B)-\Pbb(A)$; inclusion–exclusion:
$\Pbb(A\cup B)=\Pbb(A)+\Pbb(B)-\Pbb(A\cap B)$.

\section{Conditional Probability and Bayes' Theorem}
\begin{definition}[Conditional Probability]
For $\Pbb(B)>0$,
\[
  \Pbb(A\mid B)=\frac{\Pbb(A\cap B)}{\Pbb(B)}.
\]
\end{definition}
\noindent\textbf{Law of Total Probability:} If $\{B_i\}_{i\in I}$ is a partition with $\Pbb(B_i)>0$,
\[
  \Pbb(A)=\sum_{i}\Pbb(A\mid B_i)\Pbb(B_i).
\]
\noindent\textbf{Bayes' Theorem:}
\[
  \Pbb(B_j\mid A)=\frac{\Pbb(A\mid B_j)\Pbb(B_j)}{\sum_i \Pbb(A\mid B_i)\Pbb(B_i)}.
\]
\textit{Odds form:} Prior odds $\times$ likelihood ratio $=$ posterior odds.

\section{Independence}
\begin{definition}[Events]
Events $A,B$ are \emph{independent} if $\Pbb(A\cap B)=\Pbb(A)\Pbb(B)$.
A family $\{A_i\}_{i\in I}$ is \emph{mutually (jointly) independent} if for every finite distinct $i_1,\dots,i_k$,
\[
  \Pbb\!\left(\bigcap_{\ell=1}^k A_{i_\ell}\right)=\prod_{\ell=1}^k \Pbb(A_{i_\ell}).
\]
They are \emph{pairwise independent} if $\Pbb(A_i\cap A_j)=\Pbb(A_i)\Pbb(A_j)$ for all $i\ne j$.
\end{definition}
\textbf{Important:} Joint (mutual) independence $\Rightarrow$ pairwise independence, but not conversely. (相互独立 $\Rightarrow$ 两两独立,但反之不一定。)

\section{r.v.}
\begin{definition}[Random Variables]
A random variable (r.v.) $X:(\Omega,\mathcal{F})\to(\mathbb{R},\mathcal{B})$ is a measurable function:
$\{\,\omega: X(\omega)\le x\,\}\in\mathcal{F}$ for all $x\in\mathbb{R}$.
\end{definition}
Distribution of $X$: $F_X(x)=\Pbb(X\le x)$; for discrete $X$, pmf $p_X(x)=\Pbb(X=x)$; for continuous $X$, pdf $f_X=\frac{d}{dx}F_X$ (where it exists).


\section{Important Discrete Random Variables}
Below $k\in\{0,1,2,\dots\}$ unless stated.

\subsection*{Bernoulli$(p)$}
$\Pbb(X=1)=p,\ \Pbb(X=0)=1-p$.
\[
  \E[X]=p,\quad \Var(X)=p(1-p),\quad M_X(t)=1-p+pe^t.
\]

\subsection*{Binomial$(n,p)$}
Sum of $n$ i.i.d.\ Bernoulli$(p)$:
\[
  \Pbb(X=k)=\binom{n}{k}p^k(1-p)^{n-k},\quad
  \E[X]=np,\quad \Var(X)=np(1-p).
\]

\subsection*{Geometric$(p)$ (number of trials to first success)}
Support $\{1,2,\dots\}$:
\[
  \Pbb(X=k)=(1-p)^{k-1}p,\quad
  \E[X]=\frac{1}{p},\quad \Var(X)=\frac{1-p}{p^2}.
\]
Memoryless: $\Pbb(X>m+n\mid X>m)=(1-p)^n$.

\subsection*{Poisson$(\lambda)$}
\[
  \Pbb(X=k)=e^{-\lambda}\frac{\lambda^k}{k!},\quad
  \E[X]=\Var(X)=\lambda.
\]
Poisson thinning/superposition; Poisson limit of Binomial: if $n\to\infty$, $p\to 0$, $np\to\lambda$.

\subsection*{Discrete Uniform on $\{a,\dots,b\}$}
$p_X(k)=1/(b-a+1)$.
\[
  \E[X]=\frac{a+b}{2},\quad \Var(X)=\frac{(b-a+1)^2-1}{12}.
\]

\section{Continuous Random Variables}
\subsection*{Uniform$(a,b)$}
$f(x)=\frac{1}{b-a}$ on $(a,b)$.
\[
  \E[X]=\frac{a+b}{2},\quad \Var(X)=\frac{(b-a)^2}{12}.
\]

\subsection*{Exponential$(\lambda)$}
$f(x)=\lambda e^{-\lambda x}$ for $x\ge 0$.
\[
  \E[X]=\frac{1}{\lambda},\quad \Var(X)=\frac{1}{\lambda^2}.
\]
Memoryless: $\Pbb(X>t+s\mid X>s)=e^{-\lambda t}$.

\subsection*{Gaussian / Normal $\mathcal{N}(\mu,\sigma^2)$}
\[
  f(x)=\frac{1}{\sqrt{2\pi}\sigma}\exp\!\left(-\frac{(x-\mu)^2}{2\sigma^2}\right),\ \ 
  \E[X]=\mu,\ \Var(X)=\sigma^2.
\]
If $X\sim \mathcal{N}(\mu_1,\sigma_1^2)$ and $Y\sim \mathcal{N}(\mu_2,\sigma_2^2)$ independent, then $X+Y\sim \mathcal{N}(\mu_1+\mu_2,\ \sigma_1^2+\sigma_2^2)$.
Standardization: $Z=(X-\mu)/\sigma\sim\mathcal{N}(0,1)$.

\subsection*{Gamma$(\alpha,\beta)$ (shape $\alpha$, rate $\beta$)}
\[
  f(x)=\frac{\beta^\alpha}{\Gamma(\alpha)}x^{\alpha-1}e^{-\beta x},\ x\ge 0;\quad
  \E[X]=\frac{\alpha}{\beta},\ \Var(X)=\frac{\alpha}{\beta^2}.
\]
Sum of independent $\text{Gamma}(\alpha_i,\beta)$ with common rate is $\text{Gamma}(\sum \alpha_i,\beta)$.
Exponential is Gamma$(1,\lambda)$.

\subsection*{Beta$(a,b)$ on $(0,1)$}
\[
  f(x)=\frac{\Gamma(a+b)}{\Gamma(a)\Gamma(b)}x^{a-1}(1-x)^{b-1},\quad
  \E[X]=\frac{a}{a+b},\ \Var(X)=\frac{ab}{(a+b)^2(a+b+1)}.
\]
Conjugate prior for Bernoulli/Binomial.

\section{Joint Distributions and Independence of r.v.s}
For discrete $(X,Y)$: $p_{X,Y}(x,y)$; continuous: $f_{X,Y}(x,y)$. Marginals:
$p_X(x)=\sum_y p_{X,Y}(x,y)$ or $f_X(x)=\int f_{X,Y}(x,y)\,dy$.
\[
  X\perp Y \iff p_{X,Y}(x,y)=p_X(x)p_Y(y)\ (\text{discrete}) \quad
  \text{or}\quad f_{X,Y}(x,y)=f_X(x)f_Y(y)\ (\text{continuous}).
\]
Conditional distributions: $p_{X\mid Y}(x\mid y)=\dfrac{p_{X,Y}(x,y)}{p_Y(y)}$, \
$f_{X\mid Y}(x\mid y)=\dfrac{f_{X,Y}(x,y)}{f_Y(y)}$ (when denominators $>0$).

\section{Transforms and Common Tools (Very Short)}
Characteristic function $\varphi_X(t)=\E[e^{itX}]$ always exists; independence $\Rightarrow$ product of characteristic functions. CLT: sums of i.i.d.\ (finite variance) approximate normal. Chebyshev/Markov inequalities bound tails via moments.

\newpage

\section*{Exercises}

\begin{enumerate}
  \item[\textbf{1.}] \textbf{Geometric distribution properties}  
  Let $X \sim \mathrm{Geom}(p)$ be independent geometric random variables (support $\{1,2,\dots\}$, representing the trial of the first success).  Prove the \emph{memoryless property}: for any integers $m,n \geq 0$,
    \[
      \Pbb(X>m+n \mid X>m) = \Pbb(X>n).
    \]

  \item[\textbf{2.}] \textbf{Exponential distribution properties}  
  Let $X \sim \mathrm{Exp}(\lambda)$ be independent exponential random variables.  Prove the \emph{memoryless property}: for any $s,t\geq 0$,
    \[
      \Pbb(X>t+s \mid X>s) = \Pbb(X>t).
    \]

  \item[\textbf{3.}] \textbf{Closure of the Gaussian distribution}  
  Let $X\sim \mathcal{N}(\mu_1,\sigma_1^2)$ and $Y\sim \mathcal{N}(\mu_2,\sigma_2^2)$ be independent.  
  \begin{enumerate}[label=(\alph*)]
    \item Show that $X+Y \sim \mathcal{N}(\mu_1+\mu_2,\;\sigma_1^2+\sigma_2^2)$.  
    \item Generalize the result to the sum of $n$ independent Gaussian random variables.
  \end{enumerate}

  \item[\textbf{4.}] \textbf{Relationship between exponential distribution and Poisson process}  
  \begin{enumerate}[label=(\alph*)]
    \item Let $\{N(t), t\geq 0\}$ be a Poisson process with rate $\lambda$. Show that $N(t)\sim \mathrm{Poisson}(\lambda t)$.  
    \item Prove that the interarrival time between two consecutive events follows an exponential distribution $\mathrm{Exp}(\lambda)$.  
    \item Prove that interarrival times are i.i.d., establishing the link between the exponential distribution and the Poisson process.
  \end{enumerate}
\end{enumerate}

\newpage

\section*{Proof Sketches}

\subsection*{1. Geometric distribution}
(a) $X_1+X_2$ is the number of trials needed for two successes. For $k\geq 2$:
\[
  \Pbb(X_1+X_2=k) = \binom{k-1}{1} (1-p)^{k-2}p^2,
\]
which is a negative binomial $NB(r=2,p)$. In general, the sum of $n$ i.i.d.\ geometric r.v.s is $NB(r=n,p)$.  

(b) For $X\sim\mathrm{Geom}(p)$:
\[
  \Pbb(X>m+n \mid X>m) = \frac{(1-p)^{m+n}}{(1-p)^m} = (1-p)^n = \Pbb(X>n).
\]

\subsection*{2. Exponential distribution}
(a) By convolution:
\[
  f_{X_1+X_2}(s) = \int_0^s \lambda e^{-\lambda x}\, \lambda e^{-\lambda(s-x)}\,dx = \lambda^2 s e^{-\lambda s}, \quad s\ge 0,
\]
which is Gamma$(2,\lambda)$.  

(b) Memorylessness:
\[
  \Pbb(X>t+s \mid X>s) = \frac{e^{-\lambda(t+s)}}{e^{-\lambda s}} = e^{-\lambda t} = \Pbb(X>t).
\]

\subsection*{3. Gaussian distribution}
Characteristic function method:  
\[
  \varphi_X(t)=\exp\!\left(i\mu_1 t - \tfrac{1}{2}\sigma_1^2 t^2\right),\quad
  \varphi_Y(t)=\exp\!\left(i\mu_2 t - \tfrac{1}{2}\sigma_2^2 t^2\right).
\]
Thus
\[
  \varphi_{X+Y}(t)=\varphi_X(t)\varphi_Y(t)=\exp\!\left(i(\mu_1+\mu_2)t - \tfrac{1}{2}(\sigma_1^2+\sigma_2^2)t^2\right).
\]
Hence $X+Y\sim \mathcal{N}(\mu_1+\mu_2,\,\sigma_1^2+\sigma_2^2)$.  
The generalization to $n$ independent Gaussians follows similarly.

\subsection*{4. Exponential distribution and Poisson process}
(a) By definition of Poisson process: $N(t)\sim \mathrm{Poisson}(\lambda t)$.  

(b) For the first interarrival time $T_1$:
\[
  \Pbb(T_1>t)=\Pbb(N(t)=0)=e^{-\lambda t},\quad t\ge 0,
\]
so $T_1\sim \mathrm{Exp}(\lambda)$.  

(c) By the independent increments property, all interarrival times are i.i.d.\ exponential with rate $\lambda$. Thus the Poisson counting process and the exponential waiting-time distribution are two sides of the same structure.

\end{document}

\noindent\textbf{术语对照(少量)}: \quad pairwise independent = 两两独立; mutually/jointly independent = 相互独立; probability mass function (pmf) = 概率质量函数; probability density function (pdf) = 概率密度函数。
